
The   perfect phylogeny is an often used model in phylogenetics since it provides an efficient basic procedure for  representing the evolution of genomic binary characters in several frameworks, such  as for example in haplotype inference. The model, which is conceptually the  simplest, is based on the infinite sites
assumption, that is no character can mutate more than once in the whole tree. A
main open problem regarding the model is finding generalizations that retain the
computational  tractability of the original model but are more flexible in
modeling biological data when the infinite site assumption is violated because
of e.g. back mutations. A special case of back mutations that has been
considered in the study of the evolution of protein domains (where a domain is
acquired and then lost)  is persistency, that is the fact that a character is
allowed to return  back to the ancestral state.
In this model characters  can be  gained and lost at most once.
In this paper we consider the computational  problem of  explaining binary data by   the Persistent Perfect Phylogeny model
(referred as PPP) and for this purpose we investigate the problem of
reconstructing an evolution where some constraints are imposed on the paths of
the tree.


We define a  natural generalization of the PPP problem obtained by requiring
that for some pairs (character, species), neither the species nor any of its
ancestors can have the character. In other words, some characters cannot be persistent for some species.
Given a matrix $M$ and a tree $T$, how can we modify $M$ so that $T$ represents $M$?
By representing we mean perfect phylogeny, persistent phylogeny, species-driven phylogeny.

A preliminary experimental analysis shows that the
constrained persistent perfect phylogeny model allows to explain efficiently
data that do not conform with the classical perfect phylogeny model.

\begin{keyword}
\kwd{perfect phylogeny}
\kwd{persistent perfect phylogeny}
\end{keyword}


%%% Local Variables:
%%% mode: latex
%%% TeX-PDF-mode: t
%%% TeX-master: "fixed-topology-ppp.tex"
%%% buffer-file-coding-system: utf-8
%%% End:
