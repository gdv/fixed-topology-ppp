\section*{Background}

Character-based phylogeny is a broad notion to represent an evolutionary history
describing the ancestral relationships among extant taxa or individuals. Recent applications show that
the model can be applied to study the evolution of mutations related to various genomic information, such as protein domains~\cite{Pr1} or markers in tumors.
Thus in  our formulation, it is not important whether we are actually studying taxa or
individuals or other genomic data.
We will follow the usual convention of calling
\emph{species} the units of study.
The main element of this notion is that the instance is also made of
a set of \emph{characters}, and each species is in a specific state for each character~\cite{Gusfield}.
The goal is to find a phylogeny where the known species are the leaves, and the
internal nodes are labeled---just as the leaves---by a state for each character.
For each edge $(x,y)$ of the phylogeny, the mutated characters along the edge are those whose
states are different in $x$ and $y$. The simplest case is when all characters are binary, that is only two states
($0$ and $1$) are possible,
modeling the situation  when each species has or not a given feature, such as
wings (a phenotypical trait) or the mutation encoding lactase persistence (a
genotypical trait).

Moreover, we are assuming a coalescent model, that is the fact that a characteristic
shared by a set of species can be traced back to a single ancestral species.
Assuming that the state $1$ encodes the fact that a species has a given
character (for example, the fact that the species has acquired a given mutation), the coalescent
model implies that the phylogeny is directed.
% and the label of the root is
% $\vec{0}$, and $(ii)$ for each character $c$ there is at most an edge $(x,y)$
% where $y$ is a child of $x$, and the state of $c$ in $x$ is $0$, but it is $1$
% in $y$.
Restrictions on the type of changes from zero to one and vice versa lead to a variety of specific models~\cite{felsenstein:inferring-phylogenies}.




The perfect phylogeny is one of the most investigated coalescent models~\cite{Gusfield}.
Conceptually the  model is based on the infinite sites assumption, that is no
character can mutate more than once in the whole tree.
The binary perfect phylogeny problem has received much attention, culminating
the linear time algorithm when all data is known~\cite{Gus91} and an efficient
algorithm when the input data is incomplete~\cite{Sha}.
While the infinite sites assumption is quite restrictive, the perfect phylogeny model
turned out to be splendidly coherent within the haplotyping problem~\cite{bo4,Gus02}, where we want to distinguish the two
haplotypes present in each individual when given only genotype data.
More precisely, the interest here is in computing a set of haplotypes and a perfect phylogeny such that the haplotypes
(i) label the vertices of the perfect phylogeny and (ii) explain the input set of genotypes.
This context has been deeply studied in the last decade, giving rise to a number of   algorithms~\cite{Boniz,Gus06}.
%
Still, the perfect phylogeny model and the assumptions that have been central in
the previous decades cannot be employed without adaptations or improvements.
A first generalization in the literature allows for more states
(but keeping the infinite sites assumption).
In the general case, the problem is NP-hard~\cite{BFW92}, but it has
an algorithm parameterized by the number of
states~\cite{DBLP:journals/siamcomp/Fernandez-BacaL03,kannan1997fast}.
The special cases when there are three or four possible states have more
efficient algorithms~\cite{dress1992convex,kannan1994inferring,gysel2012constructing}.


Even allowing more states cannot explain the biological complexity of
real data, when homoplasy events (such as recurrent mutations or back
mutations) are present.
Two cases where those limitations are evident are the study of carcinogenesis
and of protein domains.
Carcinogenesis consists of the factors
and mechanisms that cause the onset of cancer; it
results from many combinations of mutations, but only a few, called progression
pathways, seem to account for most human tumors~\cite{subramanian2012inference}.
The observation that tumors are evolving cell populations leads to phylogeny-based
studies. At the same time the intrinsic nature of quickly and degenerately
proliferating cancer cells, results in a relative high amount of sites with
multiple mutations (i.e., in violations of the infinite sites assumption).
A protein domain is
a part of protein sequence and structure that can evolve independently of the
rest of the protein chain. Many proteins consist of several structural
domains, while a domain may appear in a variety of different proteins. In this case
it is quite frequent to acquire a domain and then to lose it~\cite{przytycka2006graph}.

Thus a central goal of this paper  is to find a model that is more
widely applicable than the perfect phylogeny, while retaining  its computational
efficiency (in fact, more general models such as the Dollo and the Camin-Sokal
models are NP-hard~\cite{felsenstein:inferring-phylogenies}).
The problem of constructing  phylogenies where the deviations from perfect phylogeny are  small
has been tackled under the name of near perfect
phylogeny~\cite{DBLP:journals/siamcomp/Fernandez-BacaL03} or near perfect
phylogeny  haplotyping  problems~\cite{DBLP:conf/ismb/SatyaMAPB06}.
Especially the impossibility of losing a character that has been previously
acquired is too restrictive, resulting in more elaborated models,
such as the persistent character~\cite{Pr1} and the General Character
Compatibility~\cite{DBLP:conf/isbra/ManuchPG11,mavnuch2009generalised}.

More precisely, the  Persistent Perfect Phylogeny
model~\cite{DBLP:journals/tcs/BonizzoniBDT12} allows each character to be lost
(\ie{} going from state $1$ to $0$) in at most an edge of the phylogeny, while the
General Character Compatibility imposes some restrictions on the possible
mutations (that is on the possible states labeling the endpoints of an edge),
while allowing the input data to be a set of possible states for each character
of a species.

We present some new computational problems that can model the problem
of finding the smallest subset of input data that must be removed to
obtain a dataset that is compatible with a given topology.




\subsection*{The persistent perfect phylogeny}

Our approach follows~\cite{DBLP:journals/tcs/BonizzoniBDT12} to which we refer
the reader for a detailed discussion of PPP, while we give here
only a cursory treatment.
The input of the PPP problem is an  $n \times m$ binary matrix $M$ whose
columns are associated with the set $C  = \{c_1, \ldots,
c_m\}$ of characters and whose rows are associated with the set $S = \{s_1, \dots,
s_n\}$ of species.
Then $M[i,j] = 1$ if and
only if the species $s_i$ has character $c_j$, otherwise $M[i,j] = 0$.
The character $c$ is \emph{gained} in the only edge where its state goes from $0$ to
$1$ or, more formally, in the edge $(x,y)$ such that $y$ is a child of $x$ and
the $c$ has state $0$ in $x$ and state $1$ in $y$.
In this case the edge $(x,y)$ is labeled by $c^{+}$.
Conversely, $c$ is \emph{lost} in the edge $(x,y)$ if $y$ is a child of $x$ and
the $c$ has state $1$ in $x$ and state $0$ in $y$.
In the latter case the edge $(x,y)$ is labeled by $c^{-}$.
For each character $c$, we allow at most one edge labeled by $c^{-}$~\cite{zeng,DBLP:journals/tcs/BonizzoniBDT12}.
% \vspace{.1in}
% \noindent

\begin{definition}[Persistent Perfect Phylogeny]
\label{def:persistent-perfect-phylogeny}
Let $M$ be an  $n \times m$ binary matrix.
Then a \emph{persistent perfect phylogeny}, in short  \emph{ p-pp},
for $M$ is a rooted tree $T$ such that:
\begin{enumerate}
\item each node $x$ of $T$ is labeled by a vector $l_x$ of length $m$;
\item
  the root of $T$ is labeled by a vector of all zeroes,  while  for each  node
$x$ of $T$ the value $l_x[j]\in\{0, 1\}$
  represents the state of character $c_j$ in tree $T$;
\item
  each edge $e=(v,w)$ is labeled by at least a character;
\item
  for each character $c_j$ there are at most  two  edges $e=(x,y)$ and
$e'=(u,v)$
  such that $l_x[j] \neq l_y[j]$ and $l_u[j] \neq l_v[j]$
  (representing a change in the state of $c_j$).
In that case $e$, $e'$  occur along the same path
  from the root of $T$ to a leaf of $T$; if $e$ is closer to the root than $e'$,
  then $l_x[j]=l_v[j]=0$, $l_y[j]=l_u[j]=1$, and
  the  edge $e$ is labeled $c_j^{+}$,
  while  $e'$ is labeled $c_j^{-}$;
\item
  each row $r$ of $M$ labels exactly one node $x$ of $T$.
  Moreover the vector $l_{x}$ is equal to the row $r$.
\end{enumerate}
%Then we say that $T$ \emph{realizes} matrix $M$, that is $M$ admits a p-pp tree.
%Whenever  a binary matrix $M$ admits a p-pp tree we say  that $M$ is \emph{solvable}.
\end{definition}

Let $s$ be a species and let $c$ be a character such that, in a persistent
perfect phylogeny $T$, the path from the root of $T$ to $s$ traverses an edge
labeled $c^{-}$.
Then $c$ is called persistent for $s$ in $T$.

The Persistent Perfect Phylogeny problem asks to find, if it exists, a
persistent perfect phylogeny for a given binary matrix $M$.
We can restate the PPP  problem  as a
variant  of the Incomplete Directed Perfect Phylogeny~\cite{Sha} by transforming
the complete input matrix into an incomplete matrix, called extended matrix.

\begin{definition}[Extended Matrix]
\label{def:Extended-Matrix}
Let $M$ be an instance of the PPP problem. The \emph{extended matrix}
associated with $M$ is an ${n \times 2m}$ matrix \me  over alphabet $\{0,1,?\}$
which is obtained by replacing each column $c$ of $M$ by a pair of columns
$(c^{+}, c^{-})$, where $?$ means that the value of such cell is not given.
Moreover for each   row $s$  of $M$
% \begin{enumerate}
% \item
if $M[s,c] = 1$, then  $M_e[s,c^{+}] = 1$ and $M_e[s,c^{-}] = 0$,
% \item
while if $M[s,c] = 0$, then  $M_e[s,c^{+}] = ?$ and $M_e[s,c^{-}] = ?$.
% ,
% otherwise $M_e[s,c^{+}] =0$ and $M_e[s,c^{-}] = 0$.
% \end{enumerate}
\end{definition}

% \begin{figure}[htbp]
% %\subfigure[A binary  matrix $M$ and the extended matrix of size $n \times 2m$]
% \begin{center}
% \includegraphics[width=0.90\textwidth]{M_Me}
% \end{center}
% \caption{A binary  matrix $M$ and the extended matrix of size $n \times 2m$}
% \label{fig:standard-tree2}
% \end{figure}


In this case the characters $(c^{+}, c^{-})$ are called conjugate.
Informally, the assignment of the \emph{conjugate} pair $(?,?)$ in a species
row $s$ for two conjugate characters ($c^{+}, c^{-}$) means that
character $c$ could be persistent in species $s$, \ie it is first gained and
then lost.
On the contrary, the pair $(1,0)$ means that character $c$ is only gained by
the species $s$.
% Finally, the pair $(0,0)$ means that character $c$ is never  gained  by
% the species $s$.
A \emph{completion}  of  a pair $(?,?)$  associated to a species $s$ and
characters ($c^{+}, c^{-}$) of  \me  consists
of forcing $\me [c^{+},s]=\me [c^{-},s]=0$ or  $\me [c^{+},s]=\me [c^{-},s]=1$,
while a partial \emph{completion}  \me is a completion of some of its conjugate
pairs.
% A completion is \emph{full} if all its conjugate pairs are completed.
Notice that $M$ admits a persistent phylogeny  if
and only if there exists a completion  of $M_e$ admitting a directed perfect
phylogeny~\cite{DBLP:journals/tcs/BonizzoniBDT12}.


 %\begin{comment}


%\end{comment}

A fundamental contribution of~\cite{DBLP:journals/tcs/BonizzoniBDT12}, building
upon~\cite{Sha}, is to frame the problem as a graph theory question.
We briefly recall here the two graphs that are used in the description of the algorithm.


Let $M$ be a binary matrix and let $c_{1}$, $c_{2}$ be two characters of $M$.
Then the configurations induced by the pair $( c_{1}, c_{2} )$ in $M$ is the
set of ordered pairs $( M[s,c_{1}], M[s,c_{2}])$ over all species $S$.
Two characters $c_{1}$ and $c_{2}$ of $M$ are \emph{conflicting} if and only if
the configurations induced by such pair
of columns is the set of all possible pairs $( 0,1)$, $( 1,1) $, $(1,0) $ and
$( 0,0) $.
The \emph{conflict graph}  $G_c =( C,E_{c}\subseteq C \times C)$ of a matrix $M$
has vertices $C$ and as edges the pairs $(c_{i}, c_{j})$ of conflicting characters (see Figure~\ref{fig:standard-tree}).
%
We also need some graph-theoretic definitions.
A graph without edges is called \emph{edgeless}.
A connected component is called \emph{nontrivial} if it has more than one
vertex.


%\begin{figure}[htbp]
%\begin{center}
%\includegraphics[width=0.50\textwidth]{m_Gc_p-pp_colored}
%\end{center}

%\caption{A matrix and its conflict graph}
%\label{fig:standard-tree}
%\end{figure}




The second graph used in the algorithm provides a representation of a  completion of characters of an extended matrix.
The \emph{red-black graph} $\grb=\langle V,E \rangle$  associated to an extended matrix \me
is the edge-colored graph where (i)
the vertices are the species and the conjugate pairs of \me (that is
for each two conjugate
characters $c^{+}$ and $c^{-}$, only $c$ is a vertex of \grb), (ii) a pair $(s,c)$
is a black edge iff the conjugate pairs
$c^{+}$ and $c^{-}$ are still incomplete in matrix \me and
$M_e[s,c^{+}]=1$ and
$M_e[s,c^{-}]=0$, (iii)
$(s,c)$ is a red edge iff the conjugate pairs
$c^{+}$ and $c^{-}$ are completed as $M_e[s,c^{+}]=M_e[s,c^{-}]=1$.






\begin{definition}
\label{definition:species-driven}
Let $T$ be a persistent phylogeny.
Then $T$ is called \emph{species-driven} if there exists a permutation
$(d_{1}, \ldots , d_{k})$ of all characters that are negated in $T$
such that for each $d_{i}$ there is a species $x_{i}$ whose path from
the root of $T$ to $x_{i}$ traverses the edge $d_{i}^{+}$ but does not
traverse any edge $d_{j}^{-}$ with $j \le i$.
\end{definition}


%%% Local Variables:
%%% mode: latex
%%% TeX-PDF-mode: t
%%% TeX-master: "fixed-topology-ppp.tex"
%%% buffer-file-coding-system: utf-8
%%% End:
