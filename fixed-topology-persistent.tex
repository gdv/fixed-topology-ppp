Focus: data una matrice M e un albero T, come modificare M perchè T
rappresenti M.
3 definizioni di "rappresenti": filogenesi perfetta, filogenesi
persistente e species-driven.
 
 
\section{Introduction}
\section{Background}
\section{Definitions}

Definition species-driven

Definition Minimum Column Removal.

The problem of removing some columns so that the remaining columns have a perfect phylogeny has been widely studied in the literature (cit. Tomescu) and is is usually modeled as finding an independent set of the conflict graph.

In the case of persistent phylogeny this approach cannot be used since there is no induced submatrix.

Inserire Risultati tesi Anna Paola

A similar problem can be defined when we also have a topology T in input.

Special case: the input matrix has a perfect phylogeny on T and we want a persistent phylogeny. In that case a matching on the conflict graph suffices.

Also maximization version, where we want to maximize the number of columns that are kept.

Notice that allowing only the removal of some rows does not really make sense if only leaves of T can be labeled by some species.
\section{Minimum Matrix Editing}

The problem asks for changing the fewest possible matrix entries, so that the resulting matrix has a (perfect/persistent/species-driven) phylogeny.

\section{Minimum Matrix Removal}

The problem asks for deleting the fewest possible rows/columns, so that the resulting matrix has a (perfect/persistent/species-driven) phylogeny.

